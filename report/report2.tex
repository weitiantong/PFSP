\documentclass{article}

\usepackage[utf8]{inputenc}
\usepackage[T1]{fontenc}
\usepackage[frenchb]{babel}

\usepackage{a4wide}

\usepackage{amsmath}
\usepackage{amssymb}
\usepackage{amsthm}

\usepackage{hyperref}

\title{Heuristic Optimization: implementation exercise 2}
\author{Samuel Buchet: 000447808}
\date{May 2017}

\begin{document}

\maketitle

\section{metaheuristic design}

\subsection{choices of iterative improvement}

According to the report, best improvement doesn't improve, so first improvement is chosen.
Insert is the best neighborhood, however, it is longer.

\subsection{stoping critera}

mean instances 50: 60.17067, x500: 30085.33\newline
mean instances 100: 563.8932, x500: 281946.6

\subsection{GRASP}

If the solution improve the best known, or if it is close enough, some perturbative steps are applied to the local search.
This step is called step two and the proximity to the best solution has been tuned in function of the number of time this step is executed.
The perturbative steps consist in applying random insert moves.
The number of moves is a parameter and tests have shown that this number should be quite low (about 5 changes for 50 jobs).

\subsection{better solution found}

50_20_03 \newline
1 28 24 14 48 15 34 45 19 5 18 37 11 29 31 9 39 10 25 33 30 44 2 43 7 41 20 40 49 47 38 4 42 17 16 22 6 26 8 50 13 21 32 23 46 36 35 3 12 27 \newline
591806

\end{document}
